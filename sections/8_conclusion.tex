\section{Conclusion}\label{sec:conclusion}

Challenges associated with Function-as-a-Service (FaaS) platforms, such as high
costs due to double billing, latency issues from cascading cold starts, and
inefficiencies in synchronous and asynchronous function calls, have been
addressed through novel approaches like \textsc{Fusionize}, SAND, and
\textsc{ProFaaStinate}. Our Project explored the integration of these approaches
within the open-source serverless platform Nuclio.

Our findings indicate that these approaches have the potential to enhance the
overall efficiency and resource utilization within serverless computing
platforms. However, Nuclio posed as an impractical choice of a FaaS platform due
to its instability, design decisions, and performance overheads. In general, the
potential of these approaches is limited by container orchestration systems that
are not designed for FaaS applications, like Kubernetes. We believe that these
solutions will yield better results if they are integrated on a platform
specifically engineered for them.

% - for future work within domain of such serverless platform we think ability
%   to measure power consumption for each container can be helpful
% - frameworks like Kepler or Scaphandre exist for this
% - operate on hypervisor level
% - such information can be integrated into a Profaastinate scheduler to achieve
%   optimal power usage and thus costs
% - users could possible pay per consumed Watt instead of per usage
% - can be combined with existing testbeds like Wiesner et al.'s Vessim to
%   develop carbon-aware application and systems

For future work in the domain of such serverless platform, we believe the
ability to measure power consumption for each container could prove beneficial.
There are existing frameworks such as Kepler \cite{amaral2023kepler} or
Scaphandre \footnote{Scaphandre Metrology Agent.
\url{https://github.com/hubblo-org/scaphandre}} that perform such tasks,
operating at the hypervisor level. This power consumption information could e.g.
be integrated into a \textsc{ProFaaStinate} scheduler, with the aim of
optimizing power usage and, therefore, costs. Implementing such a system could
potentially lead to users paying per Watt consumed rather than usage. This idea
could also be combined with existing testbeds, notably Wiesner et al.'s Vessim
\cite{wiesner2023sil}, to facilitate the development of carbon-aware
applications and systems.
