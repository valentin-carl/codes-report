\section{SAND}\label{sec:SAND}

In this section, the earlier mentioned paper \textsc{sand} is examined in more detail, providing a more detailed examination of its concepts and their implementation in Nuclio.

\subsection{Concept}
\textsc{Sand}, authored by Akkus et al. \cite{akkus2018sand} addresses similar limitations of conventional FaaS platforms as \textsc{Fusionize}, such as cascading 
cold starts or the overhead in remote function invocations, but offers a different approach to tackle these problems. Through the introduction of two pivotal 
techniques - application sandboxing and hierarchical message queuing, the serverless computing system \textsc{sand} presented in the paper is able to achieve superior 
performance in terms of lower latencies and enhanced resource efficiency compared to other serverless platforms.

\subsubsection{Application Sandboxing}
Contrary to the common practice of isolating each function within a container in serverless computing, \textsc{sand} implements 
isolation between applications. Viewing an application as a sequence of multiple serverless functions, in \textsc{sand} all functions of an application 
share a container, causing a weaker barrier between functions of the same application and reducing the occurrence of cold starts.

\subsubsection{Hierarchical Message Queuing}
Another measure taken to facilitate internal function communication is the integration of local message buses into applications. 
In common serverless computing systems, an internal request is treated the same way as an external request, which means a request from one function to another 
function in the same application must pass through the full end-to-end function call path. Using a message bus in each application container creates a shortcut for such requests, 
while also enabling asynchronous communication. Instead of sending an internal function request to, for example, a global controller, where each request passes through, in SAND 
an internal function request is sent to the local message bus, remaining in the containers network. Although \textsc{Sand} also introduces a global message bus as a fail-safe 
for message delivery, its implementation was deemed non-essential to the core improvements and therefore not integrated into Nuclio.

\subsection{Implementation Overview}
To improve Nuclio, isolation between applications and local message buses were the main concepts to be implemented.

\subsubsection{Isolation Between Applications}
When Nuclio is running on Kubernetes, every function gets its own pod. To implement application sandboxing we decided to isolate applications within pods, enabling multiple
 functions to coexist within a single pod. This approach differs from the original \textsc{Sand} implementation, which employed container-level sandboxing.  
 (why was container level sandboxing not an option?)


\subsubsection{Local Message Buses}
When one function in Nuclio invokes another, whether the function is of another or the same application, the request exits the current pod, passing through the dashboard. 
The dashboard then sends the request, making it enter another pod and finally reaching the target function, where it can be processed. Given that when using pod-level sandboxing, 
two functions of the same application are already located in the same pod, their communication can be heavily sped up.



\cite{akkus2018sand}